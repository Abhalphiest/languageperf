% !TEX TS-program = pdflatex
% !TEX encoding = UTF-8 Unicode

\documentclass[10pt]{article}
\usepackage[utf8]{inputenc} % set input encoding (not needed with XeLaTeX)
\usepackage{standalone} % for including other TeX files

%%% PAGE DIMENSIONS
\usepackage{geometry} % to change the page dimensions
\geometry{letterpaper} % 8.5x11 standard US letter paper
\geometry{margin=1in} % for example, change the margins to 2 inches all round

\usepackage{graphicx} % support the \includegraphics command and options
\usepackage{rotating}

\usepackage[parfill]{parskip} % Begin paragraphs with an empty line rather than an indent

\usepackage{booktabs} % for much better looking tables
\usepackage{array} % for better arrays (eg matrices) in maths
\usepackage{paralist} % very flexible & customisable lists (eg. enumerate/itemize, etc.)
\usepackage{verbatim} % adds environment for commenting out blocks of text & for better verbatim (monospace/code-style text)
\usepackage{subfig} % make it possible to include more than one captioned figure/table in a single float

%%% HEADERS & FOOTERS
\usepackage{fancyhdr} % This should be set AFTER setting up the page geometry
\pagestyle{fancy} % options: empty , plain , fancy
\renewcommand{\headrulewidth}{0pt} % customise the layout...
\lhead{}\chead{}\rhead{}
\lfoot{}\cfoot{\thepage}\rfoot{}

%%% SECTION TITLE APPEARANCE
\usepackage{sectsty}
\allsectionsfont{\sffamily\mdseries\upshape} % (See the fntguide.pdf for font help)
% (This matches ConTeXt defaults)

%%% ToC (table of contents) APPEARANCE
\usepackage[nottoc,notlof,notlot]{tocbibind} % Put the bibliography in the ToC
\usepackage[titles,subfigure]{tocloft} % Alter the style of the Table of Contents
\renewcommand{\cftsecfont}{\rmfamily\mdseries\upshape}
\renewcommand{\cftsecpagefont}{\rmfamily\mdseries\upshape} % No bold!

\title{421 Final Project}
\author{Margaret Dorsey}
\date{Fall 2017} % Activate to display a given date or no date (if empty),
         % otherwise the current date is printed 

\begin{document}
\maketitle

\documentclass[8pt, letterpaper]{standalone}
\usepackage{booktabs}
\usepackage{graphicx}
\usepackage{caption}
\usepackage{rotating}
\begin{document}
\minipage{1.08\textwidth}
\begin{sidewaysfigure}
\addtolength{\tabcolsep}{-1.0pt}
\begin{tabular}{|c|c|c|c|c|c|c|c|} 
\hline
&&&&&&&\\
Language & Intended Use & Imperative & Declarative & Functional & Object Oriented & Procedural & Logic \\
&&&&&&&\\
\hline
Ada &&&&&&&\\
Bash &&&&&&&\\
BASIC &&&&&&&\\
C &&&&&&&\\
C++ &&&&&&&\\
C\# &&&&&&&\\
FORTRAN &&&&&&&\\
Haskell &&&&&&&\\
Java &&&&&&&\\
Javascript &&&&&&&\\
Lisp &&&&&&&\\
Lua &&&&&&&\\
MATLAB &&&&&&&\\
Pascal &&&&&&&\\
Perl &&&&&&&\\
PHP &&&&&&&\\
Prolog &&&&&&&\\
Python &&&&&&&\\
R &&&&&&&\\
Ruby &&&&&&&\\
Rust &&&&&&&\\
Wolfram Language &&&&&&&\\
\hline
\end{tabular}
\end{sidewaysfigure}
% \begin{tabular}{|c|c|c|c|c|c|} 
% Language & Data Structured & Type Handling & Garbage Collected & Concurrent & Execution Type\\
% \end{tabular}

\endminipage
\end{document}

\end{document}
